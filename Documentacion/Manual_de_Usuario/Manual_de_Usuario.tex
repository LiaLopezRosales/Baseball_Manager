\documentclass{report}
\usepackage[spanish]{babel}
\usepackage[left=2.5cm, right=2.5cm, top=3cm, bottom=3cm]{geometry}
\usepackage{enumerate}
\usepackage{graphicx}
\usepackage{booktabs}
\usepackage{tabularx}
\usepackage{enumitem}
\usepackage{amsmath}
\usepackage{amsfonts}
\usepackage{float}
\usepackage{hyperref}   



\setlength{\parindent}{0pt}

\begin{document}

    \begin{titlepage}
        \centering
        {\bfseries\Huge Manual de Usuario \par}
        \vspace*{1cm}
        \vspace*{3cm}
        \vspace*{1cm}
        {\LARGE \textbf{Nombre: Gestión de campeonatos de béisbol} }
        \vfill
        {\bfseries\LARGE Autores: \par}
        {\Large Ariadna Vel\'azquez Rey  C311 \par} 
        {\Large L\'ia L\'opez Rosales  C312 \par} 
        {\Large Carlos Daniel Largacha Leal  C312 \par} 
        {\Large Gabriel Andr\'es Pla Lasa  C311 \par} 
        {\Large Raidel Miguel Cabellud Lizaso C311 \par} 
        \vfill
    \end{titlepage}

    \section*{Objetivo del proyecto}
    Este proyecto tiene como objetivo el desarrollo de una aplicación web que facilite la gestión de los 
    campeonatos de béisbol. Su propósito principal es permitir a los usuarios consultar y analizar información 
    relevante de las series y los peloteros, proporcionando herramientas que soporten la toma de decisiones 
    basada en estadísticas detalladas y actualizadas. El sistema será accesible, eficiente y adaptado a las 
    necesidades de diversos usuarios, desde administradores hasta periodistas.


    \section*{Requerimientos Técnicos}
    \begin{itemize}
        \item Dispositivos capaces de ejecutar navegadores modernos y conexiones a internet estables.
        \item En caso de tener una versión local también son necesarios más de 300 mb disponibles para la instalación.
    \end{itemize}
    
    \section*{Requerimientos de Software}
    Para ejecutarla de forma local es necesario seguir las siguientes orientaciones: 
    \begin{itemize}
        \item Tener instalado los paquetes para manejo de PostgreSQL, crear una base de datos de tipo psql y editar 
        los datos de especificación en el archivo .env de la carpeta principal.
        \item Tener instalado el compilador de Python y las librerías que aparecen en el archivo principal 
        pyproject.toml.
        \item Instalar con 'npm install' desde la dirección de archivo Baseball\_Management/src los módulos para 
        ejecutar React.
    \end{itemize}

    \section*{Forma de instalar la aplicación}
    Para ejecutarla desde una versión local hay dos maneras de hacerlo:
    \begin{enumerate}
        \item Ejecutar en la consola (con el permiso -x) el archivo run.sh.
        \item Ejecutar en la consola desde la dirección de la carpeta Baseball\_Management el comando 'npm start'.
    \end{enumerate}

    \section*{Breve explicación de cada una de las opciones del sistema}

    \section*{Breve explicación de cada una de las salidas del sistema}

\end{document}