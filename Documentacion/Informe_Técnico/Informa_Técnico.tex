\documentclass{report}
\usepackage[spanish]{babel}
\usepackage[left=2.5cm, right=2.5cm, top=3cm, bottom=3cm]{geometry}
\usepackage{enumerate}
\usepackage{graphicx}
\usepackage{booktabs}
\usepackage{tabularx}
\usepackage{enumitem}
\usepackage{amsmath}
\usepackage{amsfonts}
\usepackage{float}
\usepackage{hyperref} 
\usepackage[utf8]{inputenc}
\usepackage{multirow}
\usepackage{array}
\usepackage{geometry}
\usepackage{graphicx}
\geometry{a4paper, margin=1in}



\setlength{\parindent}{0pt}

\begin{document}

    \begin{titlepage}
        \centering
        {\bfseries\Huge Informe Técnico \par}
        \vspace*{1cm}
        \vspace*{3cm}
        \vspace*{1cm}
        {\LARGE \textbf{Nombre: Gestión de campeonatos de béisbol} }
        \vfill
        {\bfseries\LARGE Autores: \par}
        {\Large Ariadna Vel\'azquez Rey  C311 \par} 
        {\Large L\'ia L\'opez Rosales  C312 \par} 
        {\Large Carlos Daniel Largacha Leal  C312 \par} 
        {\Large Gabriel Andr\'es Pla Lasa  C311 \par} 
        {\Large Raidel Miguel Cabellud Lizaso C311 \par} 
        \vfill
    \end{titlepage}

    \section*{Diccionario de datos}    
    \subsection*{Tabla: User}
    \begin{tabular}{|>{\raggedright\arraybackslash}p{3cm}|>{\raggedright\arraybackslash}p{3cm}|>{\raggedright\arraybackslash}p{6cm}|>{\raggedright\arraybackslash}p{4cm}|}
        \hline
        \textbf{Campo} & \textbf{Tipo de Dato} & \textbf{Descripción} & \textbf{Restricciones} \\
        \hline
        id & AutoField (PK) & Identificador único del usuario. & Primary Key \\
        \hline
        email & EmailField & Correo electrónico del usuario. & Único, máximo 255 caracteres. \\
        \hline
        rol\_id & ForeignKey (Rol) & Rol asignado al usuario. & No nulo, relación con la tabla \texttt{Rol}. \\
        \hline
        TD\_id & ForeignKey (TechnicalDirector) & Director técnico asociado al usuario. & Nulo permitido, relación con \texttt{TechnicalDirector}. \\
        \hline
        password & CharField & Contraseña del usuario. & Máximo 128 caracteres. \\
        \hline
    \end{tabular}
    
    \textbf{Restricciones adicionales:}
    \begin{itemize}
        \item El campo \texttt{TD\_id} no puede ser nulo si el \texttt{rol\_id} es "Director Técnico".
    \end{itemize}
    
    ---
    
    \subsection*{Tabla: Rol}
    \begin{tabular}{|>{\raggedright\arraybackslash}p{3cm}|>{\raggedright\arraybackslash}p{3cm}|>{\raggedright\arraybackslash}p{6cm}|>{\raggedright\arraybackslash}p{4cm}|}
        \hline
        \textbf{Campo} & \textbf{Tipo de Dato} & \textbf{Descripción} & \textbf{Restricciones} \\
        \hline
        id & AutoField (PK) & Identificador único del rol. & Primary Key \\
        \hline
        type & CharField & Tipo de rol (Director Técnico, Admin, Usuario). & Máximo 50 caracteres, opciones predefinidas. \\
        \hline
    \end{tabular}
    
    ---
    
    \subsection*{Tabla: TechnicalDirector}
    \begin{tabular}{|>{\raggedright\arraybackslash}p{3cm}|>{\raggedright\arraybackslash}p{3cm}|>{\raggedright\arraybackslash}p{6cm}|>{\raggedright\arraybackslash}p{4cm}|}
        \hline
        \textbf{Campo} & \textbf{Tipo de Dato} & \textbf{Descripción} & \textbf{Restricciones} \\
        \hline
        id & AutoField (PK) & Identificador único del director técnico. & Primary Key \\
        \hline
        direction\_team & ForeignKey (DirectionTeam) & Equipo de dirección asociado. & Nulo permitido, relación con \texttt{DirectionTeam}. \\
        \hline
        W\_id & ForeignKey (Worker) & Trabajador asociado al director técnico. & No nulo, relación con \texttt{Worker}. \\
        \hline
    \end{tabular}
    
    \textbf{Restricciones adicionales:}
    \begin{itemize}
        \item Clave primaria compuesta: \texttt{W\_id} e \texttt{id}.
    \end{itemize}
    
    ---
    
    \subsection*{Tabla: Worker}
    \begin{tabular}{|>{\raggedright\arraybackslash}p{3cm}|>{\raggedright\arraybackslash}p{3cm}|>{\raggedright\arraybackslash}p{6cm}|>{\raggedright\arraybackslash}p{4cm}|}
        \hline
        \textbf{Campo} & \textbf{Tipo de Dato} & \textbf{Descripción} & \textbf{Restricciones} \\
        \hline
        id & AutoField (PK) & Identificador único del trabajador. & Primary Key \\
        \hline
        P\_id & ForeignKey (Person) & Persona asociada al trabajador. & No nulo, relación con \texttt{Person}. \\
        \hline
        DT\_id & ForeignKey (DirectionTeam) & Equipo de dirección asociado. & Nulo permitido, relación con \texttt{DirectionTeam}. \\
        \hline
    \end{tabular}
    
    \textbf{Restricciones adicionales:}
    \begin{itemize}
        \item Clave primaria compuesta: \texttt{P\_id} e \texttt{id}.
    \end{itemize}
    
    ---
    
    \subsection*{Tabla: DirectionTeam}
    \begin{tabular}{|>{\raggedright\arraybackslash}p{3cm}|>{\raggedright\arraybackslash}p{3cm}|>{\raggedright\arraybackslash}p{6cm}|>{\raggedright\arraybackslash}p{4cm}|}
        \hline
        \textbf{Campo} & \textbf{Tipo de Dato} & \textbf{Descripción} & \textbf{Restricciones} \\
        \hline
        id & AutoField (PK) & Identificador único del equipo de dirección. & Primary Key \\
        \hline
        Team\_id & ForeignKey (Team) & Equipo asociado al equipo de dirección. & No nulo, relación con \texttt{Team}. \\
        \hline
    \end{tabular}
    
    ---
    
    \subsection*{Tabla: Team}
    \begin{tabular}{|>{\raggedright\arraybackslash}p{3cm}|>{\raggedright\arraybackslash}p{3cm}|>{\raggedright\arraybackslash}p{6cm}|>{\raggedright\arraybackslash}p{4cm}|}
        \hline
        \textbf{Campo} & \textbf{Tipo de Dato} & \textbf{Descripción} & \textbf{Restricciones} \\
        \hline
        id & AutoField (PK) & Identificador único del equipo. & Primary Key \\
        \hline
        name & CharField & Nombre del equipo. & Máximo 100 caracteres. \\
        \hline
        color & CharField & Color del equipo. & Máximo 50 caracteres. \\
        \hline
        initials & CharField & Iniciales del equipo. & Máximo 10 caracteres. \\
        \hline
        representative\_entity & CharField & Entidad representativa del equipo. & Máximo 100 caracteres. \\
        \hline
    \end{tabular}
    
    ---
    
    \subsection*{Tabla: Person}
    \begin{tabular}{|>{\raggedright\arraybackslash}p{3cm}|>{\raggedright\arraybackslash}p{3cm}|>{\raggedright\arraybackslash}p{6cm}|>{\raggedright\arraybackslash}p{4cm}|}
        \hline
        \textbf{Campo} & \textbf{Tipo de Dato} & \textbf{Descripción} & \textbf{Restricciones} \\
        \hline
        id & AutoField (PK) & Identificador único de la persona. & Primary Key \\
        \hline
        CI & IntegerField & Cédula de identidad de la persona. & Único. \\
        \hline
        age & IntegerField & Edad de la persona. &  \\
        \hline
        name & CharField & Nombre de la persona. & Máximo 100 caracteres. \\
        \hline
        lastname & CharField & Apellido de la persona. & Máximo 100 caracteres. \\
        \hline
    \end{tabular}
    
    ---
    
    \subsection*{Tabla: Position}
    \begin{tabular}{|>{\raggedright\arraybackslash}p{3cm}|>{\raggedright\arraybackslash}p{3cm}|>{\raggedright\arraybackslash}p{6cm}|>{\raggedright\arraybackslash}p{4cm}|}
        \hline
        \textbf{Campo} & \textbf{Tipo de Dato} & \textbf{Descripción} & \textbf{Restricciones} \\
        \hline
        id & AutoField (PK) & Identificador único de la posición. & Primary Key \\
        \hline
        name & CharField & Nombre de la posición. & Máximo 100 caracteres. \\
        \hline
    \end{tabular}
    
    ---
    
    \subsection*{Tabla: BaseballPlayer}
    \begin{tabular}{|>{\raggedright\arraybackslash}p{3cm}|>{\raggedright\arraybackslash}p{3cm}|>{\raggedright\arraybackslash}p{6cm}|>{\raggedright\arraybackslash}p{4cm}|}
        \hline
        \textbf{Campo} & \textbf{Tipo de Dato} & \textbf{Descripción} & \textbf{Restricciones} \\
        \hline
        id & AutoField (PK) & Identificador único del pelotero. & Primary Key \\
        \hline
        P\_id & ForeignKey (Person) & Persona asociada al pelotero. & No nulo, relación con \texttt{Person}. \\
        \hline
        batting\_average & FloatField & Promedio de bateo del pelotero. &  \\
        \hline
        years\_of\_experience & IntegerField & Años de experiencia del pelotero. &  \\
        \hline
        pitcher & ForeignKey (Pitcher) & Lanzador asociado al pelotero. & Nulo permitido, relación con \texttt{Pitcher}. \\
        \hline
    \end{tabular}
    
    ---
    
    \subsection*{Tabla: Pitcher}
    \begin{tabular}{|>{\raggedright\arraybackslash}p{3cm}|>{\raggedright\arraybackslash}p{3cm}|>{\raggedright\arraybackslash}p{6cm}|>{\raggedright\arraybackslash}p{4cm}|}
        \hline
        \textbf{Campo} & \textbf{Tipo de Dato} & \textbf{Descripción} & \textbf{Restricciones} \\
        \hline
        P\_id & ForeignKey (BaseballPlayer) & Pelotero asociado al lanzador. & No nulo, relación con \texttt{BaseballPlayer}. \\
        \hline
        dominant\_hand & CharField & Mano dominante del lanzador. & Opciones: 'izquierda', 'derecha', 'ambas'. \\
        \hline
        No\_games\_won & PositiveIntegerField & Número de juegos ganados. &  \\
        \hline
        No\_games\_lost & PositiveIntegerField & Número de juegos perdidos. &  \\
        \hline
        running\_average & PositiveIntegerField & Promedio de carreras permitidas. &  \\
        \hline
    \end{tabular}
    
    ---
    
    \subsection*{Tabla: BPParticipation}
    \begin{tabular}{|>{\raggedright\arraybackslash}p{3cm}|>{\raggedright\arraybackslash}p{3cm}|>{\raggedright\arraybackslash}p{6cm}|>{\raggedright\arraybackslash}p{4cm}|}
        \hline
        \textbf{Campo} & \textbf{Tipo de Dato} & \textbf{Descripción} & \textbf{Restricciones} \\
        \hline
        BP\_id & ForeignKey (BaseballPlayer) & Pelotero que participa en la serie. & No nulo, relación con \texttt{BaseballPlayer}. \\
        \hline
        series & ForeignKey (Series) & Serie en la que participa el pelotero. & No nulo, relación con \texttt{Series}. \\
        \hline
        team\_id & ForeignKey (Team) & Equipo al que pertenece el pelotero. & No nulo, relación con \texttt{Team}. \\
        \hline
    \end{tabular}
    
    \textbf{Restricciones adicionales:}
    \begin{itemize}
        \item Clave primaria compuesta: \texttt{BP\_id} y \texttt{series}.
    \end{itemize}
    
    ---
    
    \subsection*{Tabla: LineUp}
    \begin{tabular}{|>{\raggedright\arraybackslash}p{3cm}|>{\raggedright\arraybackslash}p{3cm}|>{\raggedright\arraybackslash}p{6cm}|>{\raggedright\arraybackslash}p{4cm}|}
        \hline
        \textbf{Campo} & \textbf{Tipo de Dato} & \textbf{Descripción} & \textbf{Restricciones} \\
        \hline
        id & AutoField (PK) & Identificador único de la alineación. & Primary Key \\
        \hline
        team\_id & ForeignKey (Team) & Equipo asociado a la alineación. & No nulo, relación con \texttt{Team}. \\
        \hline
    \end{tabular}
    
    \textbf{Restricciones adicionales:}
    \begin{itemize}
        \item Clave primaria compuesta: \texttt{id} y \texttt{team\_id}.
    \end{itemize}
    
    ---
    
    \subsection*{Tabla: PlayerInLineUp}
    \begin{tabular}{|>{\raggedright\arraybackslash}p{3cm}|>{\raggedright\arraybackslash}p{3cm}|>{\raggedright\arraybackslash}p{6cm}|>{\raggedright\arraybackslash}p{4cm}|}
        \hline
        \textbf{Campo} & \textbf{Tipo de Dato} & \textbf{Descripción} & \textbf{Restricciones} \\
        \hline
        line\_up & ForeignKey (LineUp) & Alineación a la que pertenece el jugador. & No nulo, relación con \texttt{LineUp}. \\
        \hline
        player\_in\_position & ForeignKey (PlayerInPosition) & Jugador en una posición específica. & No nulo, relación con \texttt{PlayerInPosition}. \\
        \hline
    \end{tabular}
    
    \textbf{Restricciones adicionales:}
    \begin{itemize}
        \item Clave primaria compuesta: \texttt{line\_up} y \texttt{player\_in\_position}.
    \end{itemize}
    
    ---
    
    \subsection*{Tabla: PlayerInPosition}
    \begin{tabular}{|>{\raggedright\arraybackslash}p{3cm}|>{\raggedright\arraybackslash}p{3cm}|>{\raggedright\arraybackslash}p{6cm}|>{\raggedright\arraybackslash}p{4cm}|}
        \hline
        \textbf{Campo} & \textbf{Tipo de Dato} & \textbf{Descripción} & \textbf{Restricciones} \\
        \hline
        BP\_id & ForeignKey (BaseballPlayer) & Pelotero en una posición. & No nulo, relación con \texttt{BaseballPlayer}. \\
        \hline
        position & ForeignKey (Position) & Posición del pelotero. & No nulo, relación con \texttt{Position}. \\
        \hline
        effectiveness & FloatField & Efectividad del pelotero en la posición. &  \\
        \hline
    \end{tabular}
    
    \textbf{Restricciones adicionales:}
    \begin{itemize}
        \item Clave primaria compuesta: \texttt{BP\_id} y \texttt{position}.
    \end{itemize}
    
    ---
    
    \subsection*{Tabla: Season}
    \begin{tabular}{|>{\raggedright\arraybackslash}p{3cm}|>{\raggedright\arraybackslash}p{3cm}|>{\raggedright\arraybackslash}p{6cm}|>{\raggedright\arraybackslash}p{4cm}|}
        \hline
        \textbf{Campo} & \textbf{Tipo de Dato} & \textbf{Descripción} & \textbf{Restricciones} \\
        \hline
        id & AutoField (PK) & Identificador único de la temporada. & Primary Key \\
        \hline
        name & CharField & Nombre de la temporada. &  \\
        \hline
    \end{tabular}
    
    ---
    
    \subsection*{Tabla: Series}
    \begin{tabular}{|>{\raggedright\arraybackslash}p{3cm}|>{\raggedright\arraybackslash}p{3cm}|>{\raggedright\arraybackslash}p{6cm}|>{\raggedright\arraybackslash}p{4cm}|}
        \hline
        \textbf{Campo} & \textbf{Tipo de Dato} & \textbf{Descripción} & \textbf{Restricciones} \\
        \hline
        id & AutoField (PK) & Identificador único de la serie. & Primary Key \\
        \hline
        season & ForeignKey (Season) & Temporada a la que pertenece la serie. & No nulo, relación con \texttt{Season}. \\
        \hline
        name & CharField & Nombre de la serie. & Máximo 100 caracteres. \\
        \hline
        type & CharField & Tipo de serie. & Máximo 50 caracteres. \\
        \hline
        init\_date & DateTimeField & Fecha de inicio de la serie. &  \\
        \hline
        end\_date & DateTimeField & Fecha de finalización de la serie. &  \\
        \hline
    \end{tabular}
    
    \textbf{Restricciones adicionales:}
    \begin{itemize}
        \item Clave primaria compuesta: \texttt{season} e \texttt{id}.
        \item \texttt{init\_date} debe ser anterior a \texttt{end\_date}.
    \end{itemize}
    
    ---
    
    \subsection*{Tabla: Game}
    \begin{tabular}{|>{\raggedright\arraybackslash}p{3cm}|>{\raggedright\arraybackslash}p{3cm}|>{\raggedright\arraybackslash}p{6cm}|>{\raggedright\arraybackslash}p{4cm}|}
        \hline
        \textbf{Campo} & \textbf{Tipo de Dato} & \textbf{Descripción} & \textbf{Restricciones} \\
        \hline
        id & AutoField (PK) & Identificador único del juego. & Primary Key \\
        \hline
        local & ForeignKey (TeamOnTheField) & Equipo local. & No nulo, relación con \texttt{TeamOnTheField}. \\
        \hline
        date & DateTimeField & Fecha del juego. &  \\
        \hline
        rival & ForeignKey (TeamOnTheField) & Equipo rival. & No nulo, relación con \texttt{TeamOnTheField}. \\
        \hline
        series & ForeignKey (Series) & Serie a la que pertenece el juego. & No nulo, relación con \texttt{Series}. \\
        \hline
        score & ForeignKey (Score) & Puntuación del juego. & Nulo permitido, relación con \texttt{Score}. \\
        \hline
    \end{tabular}
    
    \textbf{Restricciones adicionales:}
    \begin{itemize}
        \item Clave primaria compuesta: \texttt{local} y \texttt{date}.
        \item \texttt{local} no puede ser igual a \texttt{rival}.
    \end{itemize}
    
    ---
    
    \subsection*{Tabla: Score}
    \begin{tabular}{|>{\raggedright\arraybackslash}p{3cm}|>{\raggedright\arraybackslash}p{3cm}|>{\raggedright\arraybackslash}p{6cm}|>{\raggedright\arraybackslash}p{4cm}|}
        \hline
        \textbf{Campo} & \textbf{Tipo de Dato} & \textbf{Descripción} & \textbf{Restricciones} \\
        \hline
        id & AutoField (PK) & Identificador único del marcador. & Primary Key \\
        \hline
        winner & ForeignKey (Team) & Equipo ganador. & No nulo, relación con \texttt{Team}. \\
        \hline
        loser & ForeignKey (Team) & Equipo perdedor. & No nulo, relación con \texttt{Team}. \\
        \hline
        w\_points & PositiveIntegerField & Puntos del equipo ganador. &  \\
        \hline
        l\_points & PositiveIntegerField & Puntos del equipo perdedor. &  \\
        \hline
    \end{tabular}
    
    \textbf{Restricciones adicionales:}
    \begin{itemize}
        \item \texttt{winner} no puede ser igual a \texttt{loser}.
        \item \texttt{w\_points} debe ser mayor o igual a \texttt{l\_points}.
    \end{itemize}
    
    ---
    
    \subsection*{Tabla: PlayerSwap}
    \begin{tabular}{|>{\raggedright\arraybackslash}p{3cm}|>{\raggedright\arraybackslash}p{3cm}|>{\raggedright\arraybackslash}p{6cm}|>{\raggedright\arraybackslash}p{4cm}|}
        \hline
        \textbf{Campo} & \textbf{Tipo de Dato} & \textbf{Descripción} & \textbf{Restricciones} \\
        \hline
        id & AutoField (PK) & Identificador único del cambio. & Primary Key \\
        \hline
        old\_player & ForeignKey (BaseballPlayer) & Jugador reemplazado. & No nulo, relación con \texttt{BaseballPlayer}. \\
        \hline
        date & DateTimeField & Fecha del cambio. &  \\
        \hline
        new\_player & ForeignKey (BaseballPlayer) & Jugador que entra. & No nulo, relación con \texttt{BaseballPlayer}. \\
        \hline
        position & ForeignKey (Position) & Posición del cambio. & No nulo, relación con \texttt{Position}. \\
        \hline
        game\_team & ForeignKey (TeamOnTheField) & Equipo en el campo. & No nulo, relación con \texttt{TeamOnTheField}. \\
        \hline
    \end{tabular}
    
    \textbf{Restricciones adicionales:}
    \begin{itemize}
        \item Clave primaria compuesta: \texttt{old\_player} y \texttt{date}.
    \end{itemize}
    
    ---
    
    \subsection*{Tabla: StarPlayer}
    \begin{tabular}{|>{\raggedright\arraybackslash}p{3cm}|>{\raggedright\arraybackslash}p{3cm}|>{\raggedright\arraybackslash}p{6cm}|>{\raggedright\arraybackslash}p{4cm}|}
        \hline
        \textbf{Campo} & \textbf{Tipo de Dato} & \textbf{Descripción} & \textbf{Restricciones} \\
        \hline
        id & AutoField (PK) & Identificador único del jugador estrella. & Primary Key \\
        \hline
        series & ForeignKey (Series) & Serie en la que es jugador estrella. & No nulo, relación con \texttt{Series}. \\
        \hline
        position & ForeignKey (Position) & Posición del jugador estrella. & No nulo, relación con \texttt{Position}. \\
        \hline
        BP\_id & ForeignKey (BaseballPlayer) & Pelotero estrella. & No nulo, relación con \texttt{BaseballPlayer}. \\
        \hline
    \end{tabular}
    
    \textbf{Restricciones adicionales:}
    \begin{itemize}
        \item Clave primaria compuesta: \texttt{series} y \texttt{position}.
    \end{itemize}
    
    ---
    
    \subsection*{Tabla: TeamOnTheField}
    \begin{tabular}{|>{\raggedright\arraybackslash}p{3cm}|>{\raggedright\arraybackslash}p{3cm}|>{\raggedright\arraybackslash}p{6cm}|>{\raggedright\arraybackslash}p{4cm}|}
        \hline
        \textbf{Campo} & \textbf{Tipo de Dato} & \textbf{Descripción} & \textbf{Restricciones} \\
        \hline
        id & AutoField (PK) & Identificador único del equipo en el campo. & Primary Key \\
        \hline
        lineup\_id & ForeignKey (LineUp) & Alineación asociada al equipo en el campo. & No nulo, relación con \texttt{LineUp}. \\
        \hline
    \end{tabular}
    
    ---
    
    \section*{Esquema de las clases definidas}
    \begin{figure}[h]
        \centering
        \includegraphics[width=0.8\textwidth]{Copy of Gestion_de_campeonatos_de_beisbol.png}
        \caption{Diagrama de gestión de campeonatos de béisbol}
        \label{fig:gestion_campeonatos}
    \end{figure}
    La aplicación sigue un flujo cliente-servidor, donde el frontend (React) y el backend (Django Rest Framework) se comunican a través de solicitudes HTTP. A continuación, se describe el flujo de información y procesamiento en cada capa de la aplicación:

    \section*{Frontend (React)}
    El frontend es responsable de la interfaz de usuario (UI) y la interacción con el usuario. Está construido con React, una biblioteca de JavaScript que permite crear interfaces dinámicas y reactivas. El flujo en el frontend es el siguiente:
    
    \subsection*{Interfaz de Usuario (UI)}
    \begin{itemize}
        \item React renderiza componentes que representan la interfaz gráfica de la aplicación.
        \item Los componentes pueden ser formularios, listas, tablas, botones, etc., que permiten al usuario interactuar con la aplicación.
    \end{itemize}
    
    \subsection*{Gestión del Estado}
    \begin{itemize}
        \item React utiliza un sistema de estado (por ejemplo, con \texttt{useState}, \texttt{useReducer} o bibliotecas como Redux) para manejar los datos que se muestran en la interfaz.
        \item El estado se actualiza dinámicamente en respuesta a las acciones del usuario (por ejemplo, hacer clic en un botón o enviar un formulario).
    \end{itemize}
    
    \subsection*{Comunicación con el Backend}
    \begin{itemize}
        \item Cuando el usuario realiza una acción que requiere datos del servidor (por ejemplo, cargar una lista de equipos o enviar un formulario), React realiza una solicitud HTTP (GET, POST, PUT, DELETE) al backend a través de la API REST.
        \item Para realizar estas solicitudes, se utiliza \texttt{fetch} o bibliotecas como Axios.
    \end{itemize}
    
    \subsection*{Renderizado Dinámico}
    \begin{itemize}
        \item Una vez que el backend responde con los datos solicitados, React actualiza el estado de la aplicación y re-renderiza los componentes para reflejar los cambios en la interfaz.
    \end{itemize}
    
    \section*{Backend (Django Rest Framework)}
    El backend es responsable de la lógica de negocio, el procesamiento de datos y la comunicación con la base de datos. Está construido con Django Rest Framework (DRF), que proporciona herramientas para crear APIs RESTful de manera eficiente. El flujo en el backend es el siguiente:
    
    \subsection*{Solicitud HTTP}
    \begin{itemize}
        \item Cuando el frontend realiza una solicitud HTTP (por ejemplo, una petición GET para obtener datos o POST para enviar datos), Django recibe la solicitud y la procesa.
    \end{itemize}
    
    \subsection*{Enrutamiento (URLs)}
    \begin{itemize}
        \item Django utiliza un sistema de enrutamiento (definido en el archivo \texttt{urls.py}) para dirigir la solicitud al ViewSet o Vista correspondiente.
    \end{itemize}
    
    \subsection*{Vistas (Viewsets)}
    \begin{itemize}
        \item Los ViewSets (o Vistas) son responsables de manejar la lógica de la solicitud. En DRF, los ViewSets permiten realizar operaciones CRUD (Crear, Leer, Actualizar, Eliminar) de manera simplificada.
        \item Las vistas interactúan con los serializadores para validar y transformar los datos entrantes o salientes.
    \end{itemize}
    
    \subsection*{Serializadores}
    Los serializadores en DRF son responsables de:
    \begin{itemize}
        \item Validar los datos enviados por el frontend (por ejemplo, datos de un formulario).
        \item Convertir los objetos de Django (modelos) en formatos como JSON para enviarlos al frontend.
        \item Convertir los datos JSON recibidos del frontend en objetos de Django para su procesamiento.
    \end{itemize}
    
    \subsection*{Modelos y Base de Datos}
    \begin{itemize}
        \item Los modelos de Django representan la estructura de la base de datos. Cada modelo define una tabla y sus campos.
        \item Cuando es necesario acceder o modificar datos, las vistas interactúan con los modelos a través del ORM (Object-Relational Mapping) de Django.
        \item El ORM permite realizar consultas a la base de datos sin escribir SQL directamente, lo que simplifica el manejo de datos.
    \end{itemize}
    
    \subsection*{Repositorios}
    \begin{itemize}
        \item Se utilizan repositorios para abstraer la lógica de acceso a datos. Los repositorios actúan como una capa intermedia entre las vistas y los modelos, centralizando las operaciones de la base de datos.
    \end{itemize}
    
    \subsection*{Respuesta HTTP}
    \begin{itemize}
        \item Una vez que la vista procesa la solicitud y obtiene los datos necesarios, devuelve una respuesta HTTP al frontend. Esta respuesta suele ser un objeto JSON que contiene los datos solicitados o un mensaje de confirmación.
    \end{itemize}
    
    \section*{Flujo Completo de una Solicitud}
    A continuación, se describe el flujo completo de una solicitud típica en la aplicación:
    
    \begin{enumerate}
        \item \textbf{Usuario realiza una acción en el frontend:}
        \begin{itemize}
            \item Por ejemplo, el usuario hace clic en un botón para cargar una lista de equipos.
        \end{itemize}
        \item \textbf{React realiza una solicitud HTTP:}
        \begin{itemize}
            \item React utiliza Axios o \texttt{fetch} para enviar una solicitud GET al endpoint correspondiente en el backend (por ejemplo, \texttt{/api/teams/}).
        \end{itemize}
        \item \textbf{Django recibe la solicitud:}
        \begin{itemize}
            \item Django dirige la solicitud al ViewSet correspondiente a través del archivo \texttt{urls.py}.
        \end{itemize}
        \item \textbf{ViewSet procesa la solicitud:}
        \begin{itemize}
            \item El ViewSet utiliza el serializador para obtener los datos de la base de datos a través del modelo.
            \item El serializador convierte los objetos de Django en JSON.
        \end{itemize}
        \item \textbf{Respuesta al frontend:}
        \begin{itemize}
            \item El ViewSet devuelve una respuesta HTTP con los datos en formato JSON.
        \end{itemize}
        \item \textbf{React actualiza el estado y la interfaz:}
        \begin{itemize}
            \item React recibe la respuesta, actualiza el estado de la aplicación y re-renderiza los componentes para mostrar los datos al usuario.
        \end{itemize}
    \end{enumerate}
    \section*{Esquema con el diseño de la aplicación}
    El diseño de la aplicación de Gestión de Campeonatos de Béisbol se fundamenta en la interacción 
    entre los distintos actores y los casos de uso definidos a continuación. \\ 

    Los actores principales de la aplicación son:
    \begin{itemize}
        \item \textbf{Administrador:} Responsable de gestionar los datos de series, equipos y jugadores, además de configurar el sistema y generar reportes.
        \item \textbf{Director Técnico:} Encargado de gestionar las alineaciones de los equipos y consultar estadísticas específicas.
        \item \textbf{Periodista/Usuario General:} Consulta las estadísticas y genera reportes basados en los datos almacenados en el sistema.
    \end{itemize}

    Y entre los principales casos de usos se encuentran:
    \begin{itemize}
        \item \textbf{Registrar datos en el sistema}
        \begin{itemize}
            \item \textbf{Actor:} Administrador
            \item \textbf{Descripción:} El administrador registra datos de temporadas, series, equipos y jugadores mediante formularios en la aplicación web.
        \end{itemize}
    
        \item \textbf{Gestionar alineaciones de equipos}
        \begin{itemize}
            \item \textbf{Actores:} Director Técnico y Administrador
            \item \textbf{Descripción:} Permite al director técnico modificar la posición de los jugadores en la alineación inicial y realizar ajustes durante una serie.
        \end{itemize}
    
        \item \textbf{Consultar estadísticas de jugadores y equipos}
        \begin{itemize}
            \item \textbf{Actores:} Administrador, Director Técnico, Periodista/Usuario General
            \item \textbf{Descripción:} El sistema permite acceder a estadísticas detalladas, como promedios de bateo, efectividad de los lanzadores y resultados de los equipos.
        \end{itemize}
    
        \item \textbf{Obtener reportes personalizados}
        \begin{itemize}
            \item \textbf{Actores:} Administrador, Director Técnico, Periodista/Usuario General
            \item \textbf{Descripción:} Genera reportes que resumen el desempeño de jugadores y equipos en una serie.
        \end{itemize}
    
        \item \textbf{Visualizar datos tabulares y gráficos}
        \begin{itemize}
            \item \textbf{Actores:} Administrador, Director Técnico, Periodista/Usuario General
            \item \textbf{Descripción:} Presentación visual de estadísticas mediante tablas y gráficos para un análisis claro y rápido.
        \end{itemize}
    
        \item \textbf{Listar equipos por clasificación}
        \begin{itemize}
            \item \textbf{Actores:} Administrador, Director Técnico, Periodista/Usuario General
            \item \textbf{Descripción:} Permite obtener la clasificación de los equipos según los resultados de cada serie.
        \end{itemize}
    
        \item \textbf{Modificar roles de usuarios}
        \begin{itemize}
            \item \textbf{Actores:} Administrador
            \item \textbf{Descripción:} El administrador puede asignar o modificar roles para los usuarios del sistema.
        \end{itemize}
    
        \item \textbf{Exportar reportes a formato PDF}
        \begin{itemize}
            \item \textbf{Actores:} Administrador, Director Técnico, Periodista/Usuario General
            \item \textbf{Descripción:} Funcionalidad para guardar reportes generados en formato PDF, con soporte para otros formatos adicionales.
        \end{itemize}
    \end{itemize}
    

    \begin{figure}[h]
        \centering
        \includegraphics[width=0.8\textwidth]{diagrama.png}
        \caption{Diagrama de casos de uso}
        \label{fig:gestion_campeonatos1}
    \end{figure}
    \section*{Solución propuesta}
    El presente sistema ha sido diseñado para optimizar la gestión de campeonatos de béisbol, 
    facilitando el registro y administración de datos clave como equipos, jugadores, alineaciones y 
    estadísticas de desempeño. Con el objetivo de mejorar la eficiencia en la organización y análisis 
    de la información, la solución implementa una aplicación web con una interfaz intuitiva y herramientas 
    avanzadas de procesamiento de datos.\\

    La propuesta se fundamenta en una arquitectura N-Capas combinada con un Microkernel, permitiendo escalabilidad y 
    modularidad en la funcionalidad del sistema. A través de un enfoque estructurado, se garantiza la seguridad y consistencia de 
    la información, ofreciendo a cada actor del sistema (administradores, directores técnicos y periodistas) las herramientas necesarias 
    para interactuar con los datos de manera eficiente.\\

    Las capas propuestas son:

    \begin{enumerate}
        \item Capa de Presentación: Interfaz de Usuario
        
        La capa de presentación es el punto de contacto entre el sistema y los usuarios. En este caso, incluiría 
        formularios, vistas y \textit{dashboards} personalizados según los roles del usuario.
        
        \begin{itemize}
            \item \textbf{Responsabilidades:}
            \begin{itemize}
                \item Diseñar formularios dinámicos para el ingreso de datos por parte del administrador.
                \item Ofrecer vistas diferenciadas para administradores, directores técnicos y periodistas.
                \item Implementar filtros interactivos para que los usuarios consulten estadísticas y generen 
                reportes personalizados.
                \item Mostrar datos tabulares, como resultados históricos y estadísticas en tiempo real.
            \end{itemize}
        \end{itemize}
        
        \item Capa de Negocio: Lógica del Sistema
        
        La capa de negocio maneja toda la lógica que regula el funcionamiento del sistema, separando las reglas de 
        negocio de las funcionalidades específicas de la interfaz o los datos.
        
        \begin{itemize}
            \item \textbf{Responsabilidades:}
            \begin{itemize}
                \item Cálculo de estadísticas avanzadas como promedios de bateo o efectividad de lanzadores.
                \item Validación de alineaciones en tiempo real para garantizar la coherencia de los equipos.
                \item Gestión de roles y permisos, asegurando que cada usuario acceda únicamente a sus funcionalidades 
                específicas.
                \item Coordinación de las solicitudes de reportes y actualización de datos en la base de datos.
            \end{itemize}
            \item \textbf{Importancia:} 
            Esta capa es el núcleo funcional del sistema, asegurando que las operaciones se realicen de manera 
            eficiente y coherente.
        \end{itemize}

        \item Capa de Acceso a Datos: Persistencia
        
        La capa de acceso a datos es responsable de interactuar directamente con la base de datos. Esto incluye la 
        consulta, modificación y almacenamiento de datos relevantes para el proyecto.
        
        \begin{itemize}
            \item \textbf{Responsabilidades:}
            \begin{itemize}
                \item Almacenar datos históricos y en tiempo real, como estadísticas de jugadores, resultados de 
                partidos y alineaciones.
                \item Proveer acceso a consultas complejas, como listar los equipos ganadores por temporada o generar 
                reportes sobre las series con mayor cantidad de juegos.
                \item Optimizar las operaciones de base de datos para soportar consultas frecuentes y minimizar 
                tiempos de respuesta.
            \end{itemize}
        \end{itemize}
    \end{enumerate}

    Y a su vez la implementación de Microkernel se encargaría:

    \begin{itemize}
        \item \textbf{Generación de Reportes:}
        \begin{itemize}
            \item Los \textit{plugins} generarían reportes en diferentes formatos, comenzando por PDF y con capacidad para añadir formatos como Excel o CVC.
        \end{itemize}

        \item \textbf{Carga Dinámica de Plugins:}
        \begin{itemize}
            \item Los \textit{plugins} se detectan en tiempo de ejecución, permitiendo al sistema integrarlos sin interrupciones.
        \end{itemize}

        \item \textbf{Integración con N-Capas:}
        \begin{itemize}
            \item La capa de negocio gestiona las solicitudes de reportes, mientras que el núcleo del Microkernel se comunica con los \textit{plugins} para entregar los resultados.
        \end{itemize}
    \end{itemize}

    Además, se han seleccionado patrones específicos para mejorar aspectos clave como la gestión de datos, la comunicación entre capas y la presentación de información. 
    Estos patrones proporcionan una estructura flexible que facilita la evolución del sistema y permite futuras mejoras sin comprometer su estabilidad. \\

    Entre los patrones implementados se encuentran:

    \begin{itemize}
        \item \textbf{Repository Pattern} el cual que actúa como una capa intermedia entre la lógica de negocio y el acceso a datos. Entre sus principales caracteristicas se ecuentran:
        \begin{enumerate}
            \item \textbf{Abstracción del acceso a datos:}
            \begin{itemize}
                \item La capa de repositorio encapsula la lógica de consulta, lo que permite desacoplar los modelos y el acceso a la base de datos de la lógica de negocio.
                \item Esto facilita realizar cambios en la estructura de la base de datos o en el mecanismo de almacenamiento sin afectar el resto del sistema.
            \end{itemize}

            \item \textbf{Interfaz unificada:}
            \begin{itemize}
                \item Proporciona una API consistente para las operaciones CRUD (Crear, Leer, Actualizar, Eliminar) y consultas personalizadas.
            \end{itemize}

            \item \textbf{Centralización de la lógica de acceso a datos:}
            \begin{itemize}
                \item Todo el código relacionado con la interacción con la base de datos está centralizado, haciendo que sea más mantenible.
            \end{itemize}

            \item \textbf{Facilidad de pruebas:}
            \begin{itemize}
                \item Como la lógica de acceso a datos está encapsulada, es posible usar simulaciones (mocks) en las pruebas unitarias para probar la lógica de negocio sin depender de la base de datos real.
            \end{itemize}
        \end{enumerate}

        \item \textbf{MVC Pattern} el cual se encuentra distribuido entre un cliente desarrollado en React y un servidor desarrollado en Django, siguiendo un enfoque RESTful. Esto implica una separación clara entre las responsabilidades del frontend y del backend:

        En el backend desarrollado con Django, los elementos del patrón MVC se distribuyen de la siguiente manera:

        \begin{itemize}
            \item \textbf{Model:} Define los modelos de datos y la lógica de negocio asociada. Cada clase de modelo representa una tabla en la base de datos y permite realizar operaciones CRUD mediante el ORM.
            
            \item \textbf{View:} En una aplicación RESTful, las vistas de Django (a través de Django REST Framework) funcionan como una capa de presentación de datos en formato JSON, no como vistas tradicionales en MVC que generan HTML.

            \item \textbf{Controller:} En el contexto de Django, el rol de ``Controlador'' en el patrón MVC es desempeñado por las vistas (\texttt{views.py}). Estas se encargan de:
                \begin{itemize}
                    \item \textit{Recibir solicitudes HTTP:} Las vistas manejan solicitudes del cliente como \texttt{GET}, \texttt{POST}, \texttt{PUT} o \texttt{DELETE}.
                    \item \textit{Interactuar con el Modelo:} Realizan operaciones sobre los datos, como consultas o actualizaciones, utilizando los modelos definidos en el ORM de Django.
                    \item \textit{Generar una Respuesta:} Transforman los datos en una respuesta adecuada (por ejemplo, JSON) que es enviada al cliente.
                \end{itemize}
        \end{itemize}

        En el frontend desarrollado con React, los elementos del patrón MVC se adaptan de la siguiente manera:

        \begin{itemize}
            \item \textbf{View:} Renderiza la interfaz de usuario (UI) en el navegador del usuario, basándose en los datos obtenidos del backend. Los componentes de React representan esta capa de la arquitectura.

            \item \textbf{Controller:} En React, no hay un controlador explícito como en el MVC clásico, pero ciertas partes de una aplicación React pueden desempeñar funciones similares a un controlador como los componentes. Los componentes pueden manejar la lógica de presentación y también gestionar interacciones del usuario, como clics, envíos de formularios o cambios en el estado. Además, los componentes suelen enviar solicitudes a la API (backend) para obtener datos.
        \end{itemize}
    \end{itemize}

    \section*{Otras informaciones relevantes}



\end{document}