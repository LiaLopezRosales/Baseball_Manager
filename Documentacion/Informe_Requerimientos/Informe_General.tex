\documentclass{report}
\usepackage[spanish]{babel}
\usepackage[left=2.5cm, right=2.5cm, top=3cm, bottom=3cm]{geometry}
\usepackage{enumerate}
\usepackage{graphicx}
\usepackage{booktabs}
\usepackage{tabularx}
\usepackage{enumitem}
\usepackage{amsmath}
\usepackage{amsfonts}
\usepackage{float}
\usepackage{hyperref}   

%\geometry{
 % top=2cm,  bottom=2cm,  left=1.5cm,  right=1.5cm
%}

\setlength{\parindent}{0pt}

\begin{document}

    \begin{titlepage}
        \centering
        {\bfseries\LARGE Facultad de Matemática y Computación \par}
        \vspace*{1cm}
        {\scshape\Large Ingeniería de Software \par}
        \vspace*{3cm}
        {\scshape\Huge Informe de Especificación de Requerimientos \par}
        \vspace*{1cm}
        {\LARGE \textbf{Tema: Gestión de campeonatos de béisbol} }
        \vfill
        {\bfseries\LARGE Integrantes: \par}
        {\Large Ariadna Vel\'azquez Rey  C311 \par} 
        {\Large L\'ia L\'opez Rosales  C312 \par} 
        {\Large Carlos Daniel Largacha Leal  C312 \par} 
        {\Large Gabriel Andr\'es Pla Lasa  C311 \par} 
        {\Large Raidel Miguel Cabellud Lizaso C311 \par} 
        \vfill
    \end{titlepage}

    \begin{center}
        \section*{Introducción}
    \end{center}

        \subsection*{Propósito del Documento}

        \subsection*{Alcance del Producto}

        \subsection*{Definiciones, acrónimos y abreviaturas}

        \subsection*{Referencias}

        \subsection*{Resumen del resto del documento}


    \newpage

    \begin{center}
        \section*{Description General}
    \end{center}

        \subsection*{Perspectiva del Producto}

        \subsection*{Funciones del Producto}

        \subsection*{Características de los usuarios}

        \subsection*{Restricciones generales}

        \subsection*{Dependencias y suposiciones}

    \newpage
    
    \begin{center}
        \section*{Requerimientos Específicos}
    \end{center}

    \vspace{0.5cm}

        \subsection*{Requerimientos Funcionales}
            \begin{enumerate}
                \item Registrar datos por el administrador mediante formularios.
                \item Generar modelos tabulares y gráficos.
                \item Gestionar los roles de la base de datos que son los usuarios especiales (administrador), los 
                directores técnicos y los usuarios normales (periodista).
                \item Obtener nombres de equipos ganadores y directores técnicos en series nacionales por temporada.
                \item Obtener nombres y posiciones de jugadores del equipo "Todos Estrellas" y su efectividad por 
                serie.
                \item Obtener series con mayor y menor cantidad de juegos celebrados.
                \item Listar equipos en primer y último lugar por serie, clasificados por tipo y orden cronológico.
                \item Obtener total de juegos ganados por un lanzador y su promedio de carreras limpias permitidas.
                \item Modificar posición de un jugador en la alineación inicial de un juego específico.
                \item Obtener estadísticas de un jugador.
                \item Exportar reportes a formato PDF con soporte para la agregación de otros formatos.
                \item Presentar diferentes vistas para los distintos tipos de roles.
                \item Generar formularios para ingresar los datos a la base de datos por parte del administrador 
                (distintos formularios para las distintas tablas de la base de datos).
                \item Tener un formulario para el director técnico que le permita realizar los cambios en  la 
                alineación.
                \item Mostrar un formulario con opciones de filtrado y solicitudes para la generación de los reportes.
            \end{enumerate}

        \vspace{1cm}

        \subsection*{Requerimientos No Funcionales}

            \vspace{0.3cm}

            \subsubsection*{Usabilidad}
            \begin{itemize}
                \item Se espera que la interfaz sea capaz de mostrar gráficas.
                \item Se espera un sistema visual de filtrado para seleccionar qué datos mostrar en los reportes.
            \end{itemize}
        
            \subsubsection*{Seguridad}
            \begin{itemize}
                \item Todos los datos personales y críticos deben ser encriptados en tránsito y en almacenamiento.
                \item La autenticación y autorización deben ser seguras, permitiendo que solo los usuarios 
                autorizados tengan acceso a funcionalidades específicas.
            \end{itemize}
        
            \subsubsection*{Portabilidad y Compatibilidad}
            \begin{itemize}
                \item El diseño de la interfaz se espera que sea ajustable al tamaño de los distintos dispositivos 
                en los que puede abrirse el sitio web.
                \item Debe ser compatible con los navegadores más utilizados, asegurando que las interfaces web 
                sean responsivas y adaptativas.
            \end{itemize}
        
            \subsubsection*{Rendimiento}
            \begin{itemize}
                \item El sistema debe ser capaz de manejar múltiples solicitudes simultáneamente sin una degradación 
                significativa en el tiempo de respuesta.
            \end{itemize}
        
            \subsubsection*{Escalabilidad}
            \begin{itemize}
                \item La arquitectura debe permitir la escalabilidad horizontal y vertical. Esto significa que se 
                debe poder agregar más recursos (como servidores adicionales) para manejar un mayor número de 
                usuarios o datos sin necesidad de reestructurar significativamente el sistema.
            \end{itemize}
        
            \subsubsection*{Mantenibilidad}
            \begin{itemize}
                \item El sistema debe estar desarrollado con buenas prácticas de programación, como la documentación 
                adecuada (docstring) y código modular.
            \end{itemize}
        
            \subsubsection*{Extensibilidad}
            \begin{itemize}
                \item La arquitectura del sistema debe ser flexible para permitir la adición de nuevas funcionalidades 
                sin necesidad de grandes modificaciones en el código base.
            \end{itemize}
        
            \subsubsection*{Almacenamiento, importación y exportación de datos}
            \begin{itemize}
                \item El software deberá de almacenar todos los datos en una base de datos SQL.
                \item El software deberá ser capaz de convertir los reportes solicitados a documentos PDF.
            \end{itemize}

        \vspace{1cm}

        \subsection*{Requerimientos de Entorno}
            \begin{itemize}
                \item Para la base de datos se ha de utilizar la tecnología PostgreSQL.
                \item En cuanto al backend de la aplicación web, este de desarrollará en python con DjangoRest como 
                librería principal, omitiendo la función de admin que viene ya preparada.
                \item El frontend se realizará en React.
            \end{itemize}

    \newpage

    \section*{Anexos}

\end{document}